\documentclass[a4paper, openany]{book}
\setcounter{tocdepth}{4}
\setcounter{secnumdepth}{4}
\usepackage{graphicx}
\usepackage{amsmath,amssymb, amsfonts, geometry, float, listings, enumerate, multicol}
\usepackage{multicol, float, color, colortbl}
\usepackage{lipsum}
\usepackage{tikz, titlesec, parskip}
\usepackage{tikz,pgfplots, circuitikz}
\usepackage{graphicx}
\usepackage{subcaption}
\usepackage{spreadtab}
\usetikzlibrary{arrows, decorations.markings}
\usepackage{tikz}
\usetikzlibrary{shapes,arrows,positioning,calc}
\usepackage{xcolor, soul} 

\pgfplotsset{compat=1.5.1}
\usepgfplotslibrary{fillbetween}
\usepackage{caption, enumitem}
\usepackage{bm}
\usepackage[export]{adjustbox}
\usepackage{mathtools}
\tikzstyle{block} = [draw, fill=white, rectangle, 
    minimum height=3em, minimum width=6em]
    
\tikzstyle{vecArrow} = [thick, decoration={markings,mark=at position
   1 with {\arrow[semithick]{open triangle 90}}},
   double distance=1pt, shorten >= 5.5pt,
   preaction = {decorate},
   postaction = {draw,line width=1pt, white,shorten >= 4.5pt}]
   
\tikzstyle{innerBlue} = [semithick, blue,line width=1pt, shorten >= 4.5pt]

\tikzset{%
    block/.style={draw, fill=white, rectangle, 
            minimum height=2em, minimum width=3em},
    input/.style={inner sep=0pt},       
    output/.style={inner sep=0pt},      
    sum/.style = {draw, fill=white, circle, minimum size=2mm, node distance=1.5cm, inner sep=0pt},
    pinstyle/.style = {pin edge={to-,thin,black}}
}


\usepackage{etoc}
\usepackage{relsize}
\usepackage{systeme}
\usepackage[pagebackref=false,colorlinks,linkcolor=blue,citecolor=magenta]{hyperref}
\usepackage{wrapfig, blindtext}
\titlespacing{\section}{0pt}{10pt}{0pt}
\titlespacing{\subsection}{0pt}{10pt}{0pt}
\titlespacing{\subsubsection}{0pt}{10pt}{0pt}

\usetikzlibrary{calc,patterns,through}
\newcommand{\arcangle}{%
	\mathord{<\mspace{-9mu}\mathrel{)}\mspace{2mu}}%
}


\newcommand{\code}{\texttt} 
\newcommand*{\plogo}{\fbox{$\mathcal{SUT}$}}
\usepackage{listings}



\renewcommand{\baselinestretch}{1.2}
 \geometry{
 a4paper,
 total={170mm,257mm},
 left=20mm,
 top=20mm,
 }
 \usepackage{multicol}
\usepackage{color}
\usepackage{transparent}
\setlength{\columnseprule}{1pt}
\def\columnseprulecolor{\color{blue}}

\usepackage{fancyhdr}
\pagestyle{fancy}




\makeatletter
\renewcommand{\thesection}{%
  \ifnum\c@chapter<1 \@arabic\c@section
  \else \thechapter.\@arabic\c@section
  \fi
}
\makeatother


\usepackage{eso-pic}
               \newcommand\BackgroundIm{
               \put(0,0){
               \parbox[b][\paperheight]{\paperwidth}{%
               \vfill
               \centering
               {\transparent{0.3}
               \includegraphics[height=\paperheight,width=\paperwidth,
               keepaspectratio]{images/background.png}%
               }
               \vfill
               }}}


\begin{document}
 \AddToShipoutPicture*{\BackgroundIm}

\begin{titlepage} % Suppresses displaying the page number on the title page and the subsequent page counts as page 1
	
	\raggedleft % Right align the title page
	\rule{1pt}{\textheight} % Vertical line
	\hspace{0.05\textwidth} % Whitespace between the vertical line and title page text
	\parbox[b]{0.75\textwidth}{ % Paragraph box for holding the title page text, adjust the width to move the title page left or right on the page
		
		{\Huge\bfseries DL Course Project}\\[2\baselineskip] % Title
		{\large\textit{Simultaneous Depth and Object Detection}}\\[4\baselineskip] % Subtitle or further description
		{\Large\textsc{Mohammad Amin Alamalhoda}} \\ % Author name, lower case for consistent small caps
		{\Large\textsc{AmirReza HatamiPour}}\\
		{\Large\textsc{MohammadReza Alimohammadi}}
		
		
		\vspace{0.5\textheight} % Whitespace between the title block and the publisher
		
		{\noindent Sharif University of Technology~~\plogo}\\[\baselineskip] % Publisher and logo
		}

\end{titlepage}

\pagenumbering{roman}



{
  \hypersetup{linkcolor=black}
  \tableofcontents
}

{
  \hypersetup{linkcolor=black}
  \listoffigures
  \listoftables
}


\fancyhf{}

\fancyhead[R]{\includegraphics[width=0.05\textwidth]{Shariflogo.png} }
\fancyhead[L]{MohammadAmin \\ Alamalhoda}
\cfoot{(\space \space \space \space \textbf{\thepage}  \space \space \space)}

\renewcommand{\headrulewidth}{1pt}
\renewcommand{\footrulewidth}{1pt}


\newpage  
\mainmatter
\pagenumbering{arabic}

\section{Git and Project Dependencies}

\vspace{0.3cm}
\subsection{Git}
	\vspace{0.3cm}

This Project is open source and is published on Github. You can watch it using \href{https://github.com/MohammadAminAlamalhoda/Deep-Object}{this link}.

You can use the following bash command for cloning this project:

\begin{lstlisting}[language=bash]
  $ git clone https://github.com/MohammadAminAlamalhoda/Deep-Object
  \end{lstlisting}
  
If you don't have \code{git} installed on your device, you can use the following bash command:
\begin{itemize}
\item Linux
\begin{lstlisting}[language=bash]
  $ sudo apt-get install git
  \end{lstlisting}
  
\item MacOS

MacOS already have git installed, check its version using bash command below:

\begin{lstlisting}[language=bash]
  $ git --version
  \end{lstlisting}
If you uninstalled it, you can install it using \code{brew}:
\begin{lstlisting}[language=bash]
  $ brew install git
  \end{lstlisting}
\item Windows

You can download source code of git and make\-install it using \href{https://git-scm.com/download/win}{this link}.
\end{itemize}



\subsection{Project Dependencies}
	\vspace{0.3cm}

This project needs the following stuff in order to be compiled successfully.

\begin{itemize}
\item -

\end{itemize}

It is noteworthy to mention that Matlab isn't open source and you should buy this product for using this, but you can ask your academic institute or university to provide you a license. You can also use \href{https://www.gnu.org/software/octave/index}{GNU Octave} which is an open\-source scientific programming language and supports all the Matlab modules and scripts.


\newpage

\section{Datas}
	\vspace{0.3cm}
\subsection{Converting Datas}
	\vspace{0.3cm}
We converted the datas which were in the .mat format to the .png  for better RAM management. This way it is possible to load the images directly from the hard drive in each itteration.

\subsection{Loading Datas}
Figure \ref{fig:sample_datas} shows a sample from the dataset.





\begin{figure}[ht]
  \centering
  \begin{subfigure}[b]{0.3\linewidth}
    \includegraphics[width=\linewidth]{images/imgNo14.png}
    \caption{Image No 1}
  \end{subfigure}
  \begin{subfigure}[b]{0.3\linewidth}
    \includegraphics[width=\linewidth]{images/depthNo14.png}
    \caption{Depth No 1}
  \end{subfigure}
    \begin{subfigure}[b]{0.3\linewidth}
    \includegraphics[width=\linewidth]{images/labelNo14.png}
    \caption{Label No 1}
  \end{subfigure}
  \begin{subfigure}[b]{0.3\linewidth}
    \includegraphics[width=\linewidth]{images/imgNo12.png}
    \caption{Image No 2}
  \end{subfigure}
  \begin{subfigure}[b]{0.3\linewidth}
    \includegraphics[width=\linewidth]{images/depthNo12.png}
    \caption{Depth No 2}
  \end{subfigure}
    \begin{subfigure}[b]{0.3\linewidth}
    \includegraphics[width=\linewidth]{images/labelNo12.png}
    \caption{Label No 2}
  \end{subfigure}
  \begin{subfigure}[b]{0.3\linewidth}
    \includegraphics[width=\linewidth]{images/imgNo8.png}
    \caption{Image No 3}
  \end{subfigure}
  \begin{subfigure}[b]{0.3\linewidth}
    \includegraphics[width=\linewidth]{images/depthNo8.png}
    \caption{Depth No 3}
  \end{subfigure}
    \begin{subfigure}[b]{0.3\linewidth}
    \includegraphics[width=\linewidth]{images/labelNo8.png}
    \caption{Label No 3}
  \end{subfigure}
  \caption{Some Sample Images from The Dataset}
  \label{fig:sample_datas}
\end{figure}

As can be seen in Figure \ref{fig:sample_datas}, dataset contains image, label, and depth.

\newpage

\subsection{Data Augmentation}
	\vspace{0.3cm}
We augmented the datas by resizing to 640, random cropping an 640$\times$640 square, and random horizontal flip. Some of the Augmented Images are plotted in Figure \ref{fig:augment}.



\begin{figure}[ht]
  \centering
  \begin{subfigure}[b]{0.3\linewidth}
    \includegraphics[width=\linewidth]{images/imgNo1272.png}
    \caption{Image}
  \end{subfigure}
  \begin{subfigure}[b]{0.3\linewidth}
    \includegraphics[width=\linewidth]{images/aug_image.png}
    \caption{Augmented Image}
  \end{subfigure}
  \caption{An Image and its Augmented Version}
  \label{fig:augment}
\end{figure}










\end{document}